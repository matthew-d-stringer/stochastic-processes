\documentclass{article}
\usepackage[utf8]{inputenc}
\usepackage{graphicx}
\usepackage{fancyhdr}
\usepackage{amsfonts}
\usepackage{amsmath}
\usepackage{amsthm}
\usepackage{amssymb}
\usepackage{enumitem}
\usepackage{listings}
\usepackage[
    letterpaper,
    margin=0.4in,
    tmargin=0.7in,
    bmargin=0.7in,
    headsep=12pt,
    footskip=12pt
]{geometry}

\theoremstyle{definition}
\newtheorem*{ques}{Question}

\renewcommand{\Re}{\operatorname{Re}}
\newcommand{\diam}{\operatorname{diam}}
\newcommand{\RR}{\mathbb{R}}

% <Laplace Transform>
\usepackage{mathrsfs}
\newsavebox\foobox
\newlength{\foodim}
\newcommand{\slantbox}[2][0]{\mbox{%
        \sbox{\foobox}{#2}%
        \foodim=#1\wd\foobox
        \hskip \wd\foobox
        \hskip -0.5\foodim
        \pdfsave
        \pdfsetmatrix{1 0 #1 1}%
        \llap{\usebox{\foobox}}%
        \pdfrestore
        \hskip 0.5\foodim
}}
\def\Laplace{\slantbox[-.45]{$\mathscr{L}$}}
% </Laplace Transform>

\pagestyle{fancy}
\fancyhf{}
\rhead{Introduction to Stochastic Processes}
\lhead{Matthew Stringer}
\chead{Chapter 1}
\cfoot{\thepage}

\title{Chapter 1 Problem Set}
\author{Matthew Stringer}
\date{}

\renewcommand{\ques}[1]{\section*{Question #1.}}

\begin{document}
    \maketitle
    \ques{1.1}
    It is possible to model this process as a Markov chain since there are 
    finite states with fixed probabilities based solely on the previous state.
    \begin{align*}
        S &= \{0, 1, 2, 3, 4, 5\} \\
        P &= 
        \begin{bmatrix}
            \frac{1}{3} & \frac{2}{3} & 0           & 0           & 0           & 0           \\
            \frac{1}{3} & 0           & \frac{2}{3} & 0           & 0           & 0           \\
            \frac{1}{3} & 0           & 0           & \frac{2}{3} & 0           & 0           \\
            \frac{1}{3} & 0           & 0           & 0           & \frac{2}{3} & 0           \\
            \frac{1}{3} & 0           & 0           & 0           & 0           & \frac{2}{3} \\
            1           & 0           & 0           & 0           & 0           & 0          
        \end{bmatrix}
    \end{align*}
\end{document}